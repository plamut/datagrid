
\documentclass[a4paper, 12pt]{book}

% XXX: why utf8x and not utf8?
\usepackage[utf8x]{inputenc}   % omogoča uporabo slovenskih črk v UTF-8

\usepackage[slovene,english]{babel}  % slovenski delilni vzorci ipd.
\usepackage[pdftex]{graphicx}  % omogoča vlaganje slik različnih formatov 
\usepackage{fancyhdr}          % poskrbi, na primer, za glave strani
\usepackage{amssymb}           % dodatni simboli
\usepackage{amsmath}
\usepackage{fixltx2e}          % textsubscript etc. (XXX: used anywhere here?)
\usepackage{color}
\usepackage{array}
\usepackage{tabulary}
\usepackage{icomma}            % comma as a decimal separator

\usepackage{numprint}          % number formating
\npthousandsep{.}\npthousandthpartsep{}\npdecimalsign{,}

\usepackage[lined,plain,linesnumbered]{algorithm2e}
\usepackage{setspace}

% reduce line spacing for algorithms to save space
\usepackage{etoolbox}
\AtBeginEnvironment{algorithm}{\setstretch{1.05}\small}

% change algorithm label name
\renewcommand{\algorithmcfname}{Psevdokoda}




\renewcommand*{\familydefault}{\rmdefault}  % default font family (roman)
\renewcommand{\baselinestretch}{1.3}  % ustrezen razmik med vrsticami

\renewcommand{\arraystretch}{1.3}  % razmik med vrsticami tabel
\AtBeginEnvironment{tabular}{\small}

%oznake strani
\renewcommand{\chaptermark}[1]%
{\markboth{\MakeUppercase{\thechapter.\ #1}}{}}
\renewcommand{\sectionmark}[1]%
{\markright{\MakeUppercase{\thesection.\ #1}}}
\renewcommand{\headrulewidth}{0.5pt} \renewcommand{\footrulewidth}{0pt}
\fancyhf{}
\fancyhead[LE,RO]{\sl \thepage} \fancyhead[LO]{\sl \rightmark}
\fancyhead[RE]{\sl \leftmark}

\newcommand{\BibTeX}{{\sc Bib}\TeX}

\newcommand{\autfont}{\Large}
\newcommand{\titfont}{\LARGE\bf}
\newcommand{\newterm}{\textit}

%\newcommand{\TODO}{\textcolor{red}}
\newcommand{\TODO}[1]{\textcolor{red}{(TODO: #1)}}
\newcommand{\sub}{\textsubscript}

\newcommand{\clearemptydoublepage}{
	\newpage{\pagestyle{empty}\cleardoublepage}}
\setcounter{tocdepth}{2}	      % globina kazala

% konstrukti
\newtheorem{izrek}{Izrek}[chapter]

%\newtheorem{trditev}{Trditev}[izrek]
\newenvironment{dokaz}{\emph{Dokaz.}\ }{\hspace{\fill}{$\Box$}}



\begin{document}
\selectlanguage{slovene}
\frontmatter
\setcounter{page}{1} %
\renewcommand{\thepage}{}  % preprecimo težave s številkami strani v kazalu


%%%%%%%%%%%%%%%%%%%%%%%%%%%%%%%%%%%%%%%%
%naslovnica
\label{naslovnica}
\thispagestyle{empty}%
\begin{center}
    {\large\sc Univerza v Ljubljani\\%
      Fakulteta za računalništvo in informatiko}%
    \vskip 10em%
    {\autfont Peter Lamut\par}%
    {\titfont \TODO{Naslov diplomskega dela} \par}%
    {\vskip 2em \textsc{DIPLOMSKO DELO NA UNIVERZITETNEM ŠTUDIJU
    RAČUNALNIŠTVA IN INFORMATIKE}\par}%
    \vfill\null%
    {\large \textsc{Mentor}: doc.\ dr. Boštjan Slivnik\par}%
    {\vskip 2em \large Ljubljana 2014 \par}%
\end{center}

% prazna stran
\clearemptydoublepage


%%%%%%%%%%%%%%%%%%%%%%%%%%%%%%%%%%%%%%%%
%copyright stran
\label{copyright_page}
\thispagestyle{empty}
\vspace*{8cm}

{\small \noindent
Rezultati diplomskega dela so intelektualna lastnina avtorja in Fakultete za
ra\-ču\-nal\-niš\-tvo in informatiko Univerze v Ljubljani.
Za objavljanje ali izkoriščanje rezultatov di\-plom\-ske\-ga dela je potrebno
pisno soglasje avtorja, Fakultete za ra\-ču\-nal\-niš\-tvo in
informatiko ter mentorja.}%
\footnote{V dogovoru z mentorjem lahko kandidat diplomsko delo s pripadajočo
izvorno kodo izda tudi pod katero izmed alternativnih licenc, ki ponuja
določen del pravic vsem: npr. Creative Commons, GNU GPL. V tem primeru na to
mesto vstavite opis licence, na primer tekst \cite{licence}

 \TODO{Morda pa res raje izdaj pod katero izmed prostih licenc?}
}


\begin{center} 
\mbox{}\vfill
% TODO: a je to potrebno tukaj? kje drugje morda?
\emph{Besedilo je oblikovano z urejevalnikom besedil \LaTeX.} 
\end{center}


% prazna stran
\clearemptydoublepage


%%%%%%%%%%%%%%%%%%%%%%%%%%%%%%%%%%%%%%%%
% stran 3 med uvodnimi listi
\noindent
\TODO{Namesto te strani {\bf vstavite} original izdane teme diplomskega 
dela s podpisom mentorja in dekana ter žigom fakultete, ki ga diplomant
dvigne v študent\-skem referatu,  preden odda izdelek v vezavo!
Glej tudi sam konec Poglavja~\ref{ch2} na strani~\pageref{pp}.}

% prazna stran
\clearemptydoublepage


%%%%%%%%%%%%%%%%%%%%%%%%%%%%%%%%%%%%%%%%
% izjava o avtorstvu
\label{izjava_avtorstvo}

\vspace*{1cm}
\begin{center} 
{\Large \textbf{\sc Izjava o avtorstvu diplomskega dela}}
\end{center}

\vspace{1cm}
\noindent Spodaj podpisani Peter Lamut,
z vpisno številko \textbf{63000200}, sem avtor diplomskega dela z naslovom:
   
\vspace{0.5cm}
\emph{\TODO{naslov diplomskega dela}}

\vspace{1.5cm}
\noindent S svojim podpisom zagotavljam, da:

\begin{itemize}
	\item sem diplomsko delo izdelal samostojno pod mentorstvom 
		doc.\ dr.\ Boštjana Slivnika,

	\item so elektronska oblika diplomskega dela, naslov (slov., angl.),
	 povzetek (slov., angl.) ter ključne besede (slov., angl.) identični
	 s tiskano obliko diplomskega dela,
	
	\item soglašam z javno objavo elektronske oblike diplomskega dela
	v zbirki ``Dela FRI''.
\end{itemize}

\vspace{1cm}

\noindent V Ljubljani, dne \TODO{datum, npr. 1. aprila 2014} \hfill
Podpis avtorja:

% prazna stran
\clearemptydoublepage


%%%%%%%%%%%%%%%%%%%%%%%%%%%%%%%%%%%%%%%%
% zahvala

\label{zahvala}
\thispagestyle{empty}\mbox{}\vfill\null\it%
\TODO{Na tem mestu zapišite, komu se zahvaljujete za izdelavo diplomske
naloge. Pazite, da ne boste koga pozabili. Utegnil vam bo zameriti. Temu
se da izogniti tako, da pozabite na celo zahvalo.}
\rm\normalfont

% prazna stran
\clearemptydoublepage


%%%%%%%%%%%%%%%%%%%%%%%%%%%%%%%%%%%%%%%%
% kazalo
\label{kazalo}
\def\thepage{}% preprecimo tezave s stevilkami strani v kazalu 
\tableofcontents{}


% prazna stran
\clearemptydoublepage


%%%%%%%%%%%%%%%%%%%%%%%%%%%%%%%%%%%%%%%%
% povzetek
% TODO: napiši na koncu
\addcontentsline{toc}{chapter}{Povzetek}
\chapter*{Povzetek}

\TODO{Ključne besede:}%

\TODO{V vzorcu je predstavljen postopek priprave diplomskega dela z uporabo
okolja \LaTeX. Vaš povzetek mora sicer vsebovati približno 100 besed, ta
tukaj je odločno prekratek.}

% prazna stran
\clearemptydoublepage


%%%%%%%%%%%%%%%%%%%%%%%%%%%%%%%%%%%%%%%%
% abstract
\selectlanguage{english}
\addcontentsline{toc}{chapter}{Abstract}
\chapter*{Abstract}

\TODO{Keywords:}

\TODO{This sample document presents an approach to typesetting your BSc
thesis using \LaTeX. A proper abstract should contain around 100 words which
makes this one way too short.}

\selectlanguage{slovene}

% prazna stran
\clearemptydoublepage


%%%%%%%%%%%%%%%%%%%%%%%%%%%%%%%%%%%%%%%%
\mainmatter
\setcounter{page}{1}
\pagestyle{fancy}

\chapter{Uvod}

\section{Splošno o podatkovnih omrežjih}

\TODO{slovenski términ za datagrid pravilen?}

Podatkovno omrežje (angl. \textit{datagrid}) je množica med seboj
povezanih računalnikov na različnih geografskih lokacijah, ki uporabnikom
omogočajo nalaganje, hranjenje in medsebojno izmenjavanje datotek.
(TODO: vir definicije / ustrezno prilagodi definicijo)

\TODO{Omeni, da imaš različne topologije, da je cilj robustnost, redundanca,
itd., sklicuj se pač na vire - odstavek ali dva max. (1 stran)}

\TODO{Pa morda še kakšna slika kot primer (nariši npr. z LaTeXom - kar tako,
 za vajo)}

\section{Replikacija podatkov v podatkovnih omrežjih}

Ko uporabnik podatkovnega omrežja pošlje zahtevek za prejem neke datoteke
ali skupine datotek, se lahko pri prenašanju podatkov od glavnega strežnika
do odjemalca porabi veliko pasovne širine. Poleg tega je lahko tudi čas,
potreben za prenos, dolg. Zaradi tega je morda smiselno za posamezno
datoteko ustvariti več njenih kopij na različnih lokacijah v omrežju.
Tovrstne kopije datotek imenujemo \newterm{replike}.

Cilj ustvarjanja replik je zmanjšati porabo pasovne širine in izboljšati
odzivne čase podatkovnega omrežja (\TODO{vir, kjer sta omenjena ta dva
cilja}). Če je neka replika na voljo tudi lokalno (bliže
uporabniku, ki jo zahteva), je namreč ni potrebno vsakič znova prenašati s
strežnika, temveč se preprosto uporabi lokalna kopija.

Posamezna vozlišča (računalniki) v omrežju praviloma nimajo dovolj prostora,
da bi hranila vse replike naenkrat. Zaradi te omejitve je potrebna
strategija, na podlagi katere se odločimo, katere replike bomo hranili
na katerih vozliščih --- seveda tako, da bo zadani cilj (\TODO{številčna
referenca prej opisanega cilja?}) čim bolje izpolnjen.

Replikacijske strategije ločimo v dve skupini, in sicer poznamo
\newterm{statično replikacijo} in \newterm{dinamično replikacijo}
(\TODO{viri, kjer je to navedeno}).

\subsection{Statična replikacija}

Pri statični replikaciji za vsako posamezno vozlišče že vnaprej določimo,
katere replike bodo shranjene na njem. Problem, kako replike čim bolje
razporediti po omrežju, lahko obravnavamo kot statičen optimizacijski
problem. Zanj se sicer izkaže, da je tako NP-težek kot tudi
neaproksimativen.

\TODO{referenca kakšnega članka, ki o tem govori, npr. Čibej}

\subsection{Dinamična replikacija}

Pri dinamični replikaciji se vsako vozlišče avtonomno odloča, katere
replike bo hranilo, pri čemer se lahko množica shranjenih replik na
posameznem vozlišču skozi čas spreminja. Če vozlišče na podlagi neke
metrike oceni, da je pravkar zahtevana replika zanj bolj pomembna od ene
ali več obstoječih shranjenih replik, lahko slednje izbriše in namesto
njih shrani novo repliko.

Dinamična replikacija je boljša od statične, saj se lahko avtomatično
prilagaja morebitnim spremembam v vzorcih uporabe podatkovnega omrežja
\TODO{citat Čibej}. Poznamo veliko različnih dinamičnih replikacijskih
strategij, pri čemer je ena izmed najbolj učinkovitih algoritem
\newterm{Fast Spread} (\TODO{vir -- Ranganathan and Foster, 2001b?}).

\TODO{Fast spread prevedemo v slovenščino? ``algoritem hitrega širjenja''?}


\section{Algoritem Fast Spread}

Algoritem Fast Spread se uporablja v hierarhično urejenih podatkovnih omrežjih.
Na vrhu hierarhije je glavni strežnik, ki ima dovolj prostora, da vedno hrani
vse replike, ki obstajajo v omrežju. V primerjavi z njim je prostor na vseh
ostalih vozliščih relativno majhen, tako da lahko slednja naenkrat hranijo
zgolj omejeno podmnožico replik. Vozlišča so lahko poljubno povezana med seboj
in ni nujno, da ima vsako izmed njih neposredno povezavo z glavnim strežnikom
-- z njim so lahko povezana tudi zgolj posredno preko drugih vozlišč.

Za vsako vozlišče v omrežju obstaja najkrajša pot do strežnika in vsako izmed
njih ve, katero vozlišče je njihov starš na tej poti. Kadar vozlišče prejme
zahtevek za določeno repliko in slednje nima shranjene lokalno, zahtevek
posreduje svojemu staršu, ta pa lahko isti zahtevek posreduje še naprej
navzgor po hierarhiji, dokler ga ne prejme vozlišče, ki repliko ima. Če ne
prej, se to zgodi takrat, ko zahtevek prispe do glavnega strežnika.

\TODO{kakšna lepa slikica?}

Prvo vozlišče v verigi, ki najde zahtevano repliko pri sebi, le-to pošlje
vozlišču, ki jo je od njega zahtevalo. Slednje jo nato posreduje še naprej
navzdol po hierarhiji, dokler replika ne prispe do vozlišča, ki je poslalo
prvotni zahtevek zanjo. Pri tem vsako izmed vozlišč na poti repliko tudi
lokalno shrani.

V primeru, da določeno vozlišče nima dovolj prostora, da bi shranilo repliko,
mora predhodno izbrisati eno ali več obstoječih replik. Katere izmed njih bodo
zamenjane z novo repliko, je odvisno od tega, katero strategijo zamenjave
uporablja vozlišče. Poznamo veliko različnih strategij, npr.
\TODO{morda naštej nekaj?}. Izmed njih bom opisal dve, ki so jih avtorji
člankov \TODO{referenca?} uporabili za primerjavo s strategijama, ki so ju
predlagali sami.

\TODO{kakšna referenca na vir, ki Fast Spread opisuje?}

\subsection{Strategija zamenjave LRU}

Pri strategiji \newterm{LRU} (\newterm{``Least Recently Used"}) vozlišče
zamenja tisto repliko (ali skupino replik), pri kateri je preteklo največ
časa od prejema zadnjega zahtevka zanjo --- torej tisto, ki najdlje časa
ni bila uporabljena.

\subsection{Strategija zamenjave LFU}

Pri strategiji \newterm{LFU} (\newterm{``Least Frequently Used"}) vozlišče
zamenja tisto repliko (ali skupino replik), ki je bila v preteklosti
najmanjkrat zahtevana. Strategija torej upošteva vse pretekle zahtevke za
posamezno repliko in ne zgolj zadnjega, kot je to pri strategiji LRU.

\section{Trditev v člankih: izboljšava algoritma Fast Spread}

Leta 2011 in 2012 sta bila v dveh različnih strokovnih revijah objavljena
dva članka istih avtorjev, v katerih so slednji predstavili dve novi
strategiji zamenjave, ki naj bi po njihovih trditvah dosegali dosti boljše
rezultate od strategij LRU in LFU. Strategiji
so poimenovali \newterm{EFS} (\newterm{``Enhanced Fast Spread''}) in
\newterm{MFS} (\newterm{``Modified Fast Spread''})
\TODO{vir oz. referenca člankov}.

\TODO{oštevilči članka? Da se kasneje v besedilu sklicuješ nanj? Kako se
to ponavadi dela?}

Članka sta si med seboj že na prvi pogled zelo podobna. Oba imata identično
strukturo (identične naslove posameznih delov članka), vsebujeta številne
identične odseke besedila, identične diagrame in tudi opisa obeh strategij
v psevdokodi sta v veliki meri enaka. Na podlagi naštetega se je pojavil
resen sum na recikliranje člankov (avtoplagiatorstvo).

\TODO{vstavi slike, dele besedila itd. za primerjavo pa to omeni v tekstu}

V okviru diplomskega dela sem z implementacijo strategij, opisanih v
člankih, želel doseči predvsem naslednje cilje:
\label{cilji}

\begin{enumerate}
\item Preveriti rezultate, ki so jih dosegli avtorji člankov.

\item Neposredno primerjati strategiji iz člankov \TODO{2011} in
\TODO{2012} med seboj. V obeh člankih namreč avtorji v njih opisani
strategiji primerjajo zgolj s strategijama LRU in LFU, medsebojne
primerjave pa v novejšem članku (\TODO{2012}) niso naredili.

\item Na podlagi primerjave obeh opisanih strategij ugotoviti, ali med
člankoma sploh obstaja kakšna bistvena vsebinska razlika.
\end{enumerate}


\chapter{Primerjava strategij iz člankov}

\section{Opis strategij zamenjave}
Kot je že bilo omenjeno (\TODO{v poglacju XYZ?}), algoritem Fast Spread
shrani zahtevano repliko na vsakem vozlišču na poti od vozlišča, kjer je
bila zahtevana replika najdena, do vozlišča, ki je repliko prvotno
zahtevalo. Če določeno vozlišče nima dovolj razpoložljivega prostora,
da bi zahtevano repliko shranilo, mora najprej izbrisati eno ali več
obstoječih replik.

Avtorji člankov (\TODO{št. oznaka člankov?}) opozarjajo, da je lahko
skupina replik, ki jih je potrebno izbrisati, ``bolj pomembna" od
nove replike. Izpostavljajo, da algoritem Fast Spread tega ne upošteva in
da zamenjavo skupine obstoječih replik z novo repliko vedno
izvede ne glede na morebitno večjo vrednost skupine replik
(\TODO{sklic na ustrezen del članka - oz. v obeh}).

Avtorji v obeh člankih predlagajo nekoliko spremenjeni različici algoritma
Fast Spread, ki izvedeta zamenjavo le v primerih, ko je vrednost skupine
replik strogo manjša od vrednosti nove replike. Pri tem algoritma za
ocenjevanje vrednosti (skupin) replik uporabljata metrike, opisane v
nadaljevanju.

\subsection{Strategija EFS}

Psevdokoda strategije EFS je prikazana na Sliki~\ref{alg:EFS}, pri čemer
je pomen posameznih spremenljivk pojasnjen v Tabeli~\ref{tbl:EFS_vars}.

\begin{algorithm}
  \label{alg:EFS}
  \caption{Strategija EFS (\TODO{preimenuj bold oznako + referenca
  na članek})}

  \SetKwComment{blankln}{}{}

  Initialize SOS to $0$\;
  \eIf{RR exists on RN}{
      Use RR\;
  }{
      \For{$i = 2$ \KwTo NSPList.size}{
          \If{RR exists on NSPList(i)}{
              \For{$j = \mathit{NSPList}(i - 1)$ \KwTo $1$}{
                  \eIf{CNFSS $\geq$ RR.Size}{
                      Copy RR\;
                  }{
                      \For{$x = 1$ \KwTo ReplicaList.size}{
                          \eIf{$\mathit{SOS} + \mathit{CNFSS} <$ RR.Size}{
                              SOS = SOS + \textit{SizeList(x)}\;
                          }{
                              Break\;
                          }
                      }

\blankln{}

                      $\mathit{GV} = \frac{\sum_{y=1}^{x-1}
                          \mathit{NORList(y)}}{
                                  \sum_{y=1}^{x-1} \mathit{SizeList(y)}} +
                          \frac{\sum_{y=1}^{x-1}
                              \mathit{NORFSTIList(y)}}{\mathit{FSTI}} +
                          \frac{1}{\mathit{CT} - \frac{\sum_{y=1}^{x-1}
                              \mathit{LRTList(y)}}{x-1}}$

\blankln{}

                      $\mathit{RV_{RR}} =
                          \frac{\mathit{NORRR}}{\mathit{SRR}} +
                          \frac{\mathit{NORRRFSTI}}{\mathit{FSTI}} +
                          \frac{1}{\mathit{CT} - \mathit{LRTRR}}$

\blankln{}

                      \If{$\mathit{GV} < \mathit{RV_{RR}}$} {
                          \For{$y = 1$ \KwTo $x-1$}{
                            Delete \textit{ReplicaList(y)},
							  \textit{NORList(y)},
							  \textit{SizeList(y)}, \textit{NORFSTIList(y)},
							  \textit{LRTList(y)}\;
                          }
                          Copy RR;
                      }
                  }
              }
          }
      }
  }
\end{algorithm}

\begin{table}
\small
  \begin{center}
    \begin{tabulary}{1.0\textwidth}{ >{\itshape}p{7em} J}
      \textnormal{Spremenljivka} & Pomen \\
      \hline
      RR & zahtevana replika \\
      RN &  vozlišče, ki zahteva repliko \\
      CNFSS & nezaseden prostor na trenutno opazovanem vozlišču \\
      FSTI & frekvenčno-specifičen časovni razpon \TODO{prevod???} \\
      NORRR & število zahtevkov za \textit{RR} \\
      SRR & velikost \textit{RR} \\
      NORRRFSTI & število zahtevkov za zahtevano repliko
          v \textit{FSTI} \\
      CT & trenutni čas \\
      LRTRR & čas zadnjega zahtevka za zahtevano repliko \\
      SOS & spremenljivka, ki predstavlja vsoto velikosti skupine replik na
          trenutno opazovanem vozlišču \\
      NSPList & seznam vozlišč na najkrajši poti od \textit{RN} do
          glavnega strežnika \\
      ReplicaList & seznam obstoječih replik na trenutno opazovanem vozlišču,
          urejenih po naraščajočem vrstnem redu glede na njihovo vrednost
          \textit{RV}. Če ima več replik enako vrednost \textit{RV}, so med
          seboj urejene naključno. \\
      NORList & seznam, ki vsebuje vrednosti, kolikokrat so bile istoležne
          replike s seznama \textit{ReplicaList} zahtevane s strani
          trenutno opazovanega vozlišča. \\
      SizeList & seznam, ki vsebuje velikosti istoležnih replik s seznama
          \textit{ReplicaList} \\
      NORFSTIList & seznam, ki vsebuje vrednosti, kolikokrat so bile istoležne
          replike s seznama \textit{ReplicaList} zahtevane v \textit{FSTI} s
          strani trenutno opazovanega vozlišča \\
      LRTList & seznam, ki vsebuje čase zadnjega zahtevka za istoležne
          replike s seznama \textit{ReplicaList}
    \end{tabulary}
 \end{center}

  \caption{Pomen spremenljivk v psevdokodi algoritma EFS.}
  \label{tbl:EFS_vars}
\end{table}

Kadar neko vozlišče prejme zahtevano repliko, za shranitev katere nima dovolj
nezasedenega prostora (\TODO{ref. vrstica 10}), pogleda, koliko
obstoječih replik bi zaradi tega moralo predhodno izbrisati. Sprehodi se skozi
seznam~\textit{\mbox{ReplicaList}}, dokler vsota velikosti obiskanih replik in
velikosti nezasedenega prostora ne doseže (ali preseže) velikosti nove replike
(\TODO{referenca vrstice 11--17}). Vse obiskane replike nato implicitno
tvorijo skupino replik, ki bodo potencialno izbrisane.

Vozlišče v nadaljevanju izračuna vrednosti nove replike in skupine replik
(vrstici 18 in 19 \TODO{ref.}) ter ju med seboj primerja. V primeru, da je
vrednost skupine manjša (vrstica 20 \TODO{}), vozlišče izbriše vse replike
iz skupine in njihove statistike ter jih nadomesti z novo repliko
(vrstice 21--24 \TODO{ref.}).

Strategija EFS pri ocenjevanju vrednosti replik upošteva naslednje štiri
stvari:

\begin{itemize}
  \item število zahtevkov za repliko,
  \item pogostost zahtevkov,
  \item velikost replike,
  \item čas od zadnjega zahtevka za repliko.
\end{itemize}

\TODO{item spacing too big}

Število zahtevkov, njihova pogostost in čas od zadnjega zahtevka so pomembni
dejavniki, saj z njihovo pomočjo lahko ocenimo verjetnost, da bo replika
v prihodnosti ponovno zahtevana. Ker imajo vozlišča omejen prostor za
shranjevanje replik, je pomembna tudi velikost slednjih
\TODO{referenca na članek 2011}.

\begin{samepage}
Iz vrstice 19 (\TODO{referenca}) psevdokode \TODO{sklic} lahko razberemo
formulo za izračun vrednosti posamezne replike:
\begin{equation}
  \mathit{RV} = \frac{\mathit{NORRR}}{\mathit{SRR}} +
                      \frac{\mathit{NORRRFSTI}}{\mathit{FSTI}} +
                      \frac{1}{\mathit{CT} - \mathit{LRTRR}}
  \label{eq:EFS_RV}
\end{equation}
pri čemer je pomen posameznih spremenljivk pojasnjen v
Tabeli~\ref{tbl:EFS_vars}.
\end{samepage}

Iz prvega in drugega člena enačbe~\eqref{eq:EFS_RV} je razvidno, da imajo
višjo vrednost tiste replike, ki so bile večkrat zahtevane (večje število
zahtevkov), pri čemer drugi člen upošteva zgolj zahtevke iz nedavne zgodovine,
torej znotraj intervala FSTI, ki jih dodatno normira z velikostjo tega
intervala. Iz prvega člena je tudi razvidno, da je vrednost replik obratno
sorazmerna z njihovo velikostjo. Večje replike so vrednotene niže, saj
zasedajo več prostora na vozlišču. Zadnji člen daje višjo vrednost replikam,
katerih čas od njihovega zadnjega zahtevka je manjši.

\TODO{zdaj pa še group value pojasni}

\begin{samepage}
Podobno kot vrednost posamezne replike se izračuna tudi vrednost skupine
replik, in sicer po formuli iz vrstice 18 \TODO{} v
\TODO{EFS psevdokoda referenca}:
\begin{equation}
  \mathit{GV} = \frac{\sum\limits_{y=1}^{x-1} \mathit{NORList(y)}}{
                      \sum\limits_{y=1}^{x-1} \mathit{SizeList(y)}} +
                \frac{\sum\limits_{y=1}^{x-1} \mathit{NORFSTIList(y)}}{
                      \mathit{FSTI}} +
                \frac{1}{\mathit{CT} - \frac{\sum\limits_{y=1}^{x-1}
                         \mathit{LRTList(y)}}{x-1}}
  \label{eq:EFS_GV}
\end{equation}
pri čemer je pomen posameznih spremenljivk pojasnjen v
Tabeli~\ref{tbl:EFS_vars}.
\end{samepage}

Enačbi~\eqref{eq:EFS_RV}~in~\eqref{eq:EFS_GV} sta vsebinsko gledano
pravzaprav enaki. Edina razlika je, da so posamezni členi iz
enačbe~\eqref{eq:EFS_RV} v enačbi~\eqref{eq:EFS_GV} namesto za posamezno
repliko izračunani na ravni celotne skupine replik.
Tako je med drugim v prvem členu enačbe~\eqref{eq:EFS_GV} izračunano razmerje
med skupnim številom zahtevkov in skupno velikostjo \textit{vseh replik} iz
skupine, v tretjem členu pa je uporabljen \textit{povprečen čas} od zadnjega
zahtevka za posamezno repliko v skupini.

\TODO{a bi v GV morali deliti 2. člen še s številom replik v skupini?
Konceptualno gledano verjetno da, ampak v članku to ni tako navedeno) ---
na to opozori pri razpravi o članku}


\subsection{Strategija MFS}

Strategija MFS (psevdokoda na Sliki~\ref{alg:MFS})
je konceptualno popolnoma enaka strategiji EFS. Od nje se razlikuje
zgolj v formulah za izračun vrednosti replike in vrednosti skupine replik
ter v načinu, na katerega so replike urejene v seznamu \textit{ReplicaList}.

\begin{samepage}
Vrednost replike se izračuna po naslednji formuli iz vrstice 11
(\TODO{ref.}) v Psevdokodi~\ref{alg:MFS}:
\begin{equation}
  \mathit{PNOR} = \mathit{NOR} \times
                  \frac{\mathit{RR.Size} - \mathit{CNFSS}}{
                        \mathit{ RR.Size}}
  \label{eq:MFS_RV}
\end{equation}
pri čemer je pomen posameznih spremenljivk pojasnjen v
Tabeli~\ref{tbl:MFS_vars}.
\end{samepage}

Vrednost replike je PNOR oziroma ``delno število zahtevkov''. Izračuna
se kot skupno število zahtevkov za repliko (NOR), prilagojeno za
korekcijski faktor. Slednji je odvisen od razmerja med velikostjo replike
in velikostjo nezasedenega prostora na vozlišču. V robnem primeru,
ko je ves razpoložljiv prostor na vozlišču zaseden (CNFSS je enak 0), sta
vrednosti PNOR in NOR enaki. PNOR ni nikoli negativno število.%
\footnote{Ulomek v enačbi~\eqref{eq:MFS_RV} je vedno pozitiven, saj
je njegov imenovalec pozitiven -- velikost replike \textit{RR.Size} je
namreč strogo večja od velikosti nezasedenega prostora \textit{CNFSS}. Če
bi bilo obratno, bi namreč vozlišče repliko takoj shranilo in niti ne bi
šlo računati njene vrednosti, kar je razvidno iz vrstic 8 in 9
 \TODO{(ref.)} v Psevdokodi~\ref{alg:MFS}.}



\begin{samepage}
Vrednost skupine replik se izračuna po formuli, ki se nahaja na levi strani
neenakosti v vrstici 18 (\TODO{ref.}) Psevdokode~\ref{alg:MFS}:
\begin{equation}
	\sum\limits_{y=1}^{x-1} \mathit{NORList(y)}
  \label{eq:MFS_GV}
\end{equation}
pri čemer je pomen posameznih spremenljivk pojasnjen v
Tabeli~\ref{tbl:MFS_vars}.
\end{samepage}

Formula~\eqref{eq:MFS_GV} je zelo enostavna, saj zgolj sešteje število
zahtevkov (NOR) za vse replike, ki sestavljajo skupino.

Kot že omenjeno se strategija MFS od strategije EFS nekoliko razlikuje
tudi v tem, iz katerih replik vozlišče sestavi skupino replik, ki bodo morda
zamenjane. Razlog za to je različna urejenost replik v seznamu
\textit{ReplicaList}. Medtem ko strategija EFS replike uredi glede na
njihovo vrednost (RV), jih strategija MFS uredi glede na
število zahtevkov zanje, v primeru enakih vrednosti pa še dodatno
glede na njihovo velikost. Urejenosti seznama \textit{ReplicaList} je
sicer opisana tudi v Tabelah~\ref{tbl:EFS_vars}~oziroma~\ref{tbl:MFS_vars}.

\TODO{pseudocode vertical spacing?}

\begin{algorithm}[H]
  \label{alg:MFS}
  \caption{Strategija MFS (\TODO{preimenuj bold oznako + referenca
  na članek})}

  \SetKwComment{blankln}{}{}

  Initialize SOS to $0$\;
  \eIf{RR exists on RN}{
      Use RR\;
  }{
      \For{$i = 2$ \KwTo NSPList.size}{
          \If{RR exists on NSPList(i)}{
              \For{$j = \mathit{NSPList}(i - 1)$ \KwTo $1$}{
                  \eIf{CNFSS $\geq$ RR.Size}{
                      Copy RR\;
                  }{
                      $\mathit{PNOR} =
                          \mathit{NOR} \times
                          \tfrac{\mathit{RR.Size} - \mathit{CNFSS}}{
                                 \mathit{ RR.Size}}$
                      \For{$x = 1$ \KwTo ReplicaList.size}{
                          \eIf{SOS $<$ RR.Size $- \mathit{CNFSS}$}{
                              SOS = SOS + \textit{SizeList(x)}\;
                          }{
                              Break\;
                          }
                      }

                      \If{$\sum_{y=1}^{x-1} \mathit{NORList(y)} <
                          \mathit{PNOR}$} {
                          \For{$y = 1$ \KwTo $x-1$}{
                              Delete \textit{ReplicaList(y)},
                              \textit{SizeList(y)}, \textit{NORList(y)}\;
                          }
                          Copy RR;
                      }
                  }
              }
          }
      }
  }
\end{algorithm}


\begin{table}
\small
  \begin{center}
    \begin{tabulary}{1.0\textwidth}{ >{\itshape}p{7em} J}
      \textnormal{Spremenljivka} & Pomen \\
      \hline
      RR & zahtevana replika \\
      RN &  vozlišče, ki zahteva repliko \\
      CNFSS & nezaseden prostor na trenutno opazovanem vozlišču \\
      NOR & število zahtevkov za \textit{RR} \\
      PNOR & delno število zahtevkov za \textit{RR} \\
      SOS & spremenljivka, ki predstavlja vsoto velikosti skupine replik na
          trenutno opazovanem vozlišču \\
      NSPList & seznam vozlišč na najkrajši poti od \textit{RN} do
          glavnega strežnika \\
      ReplicaList & seznam obstoječih replik na trenutno opazovanem vozlišču,
          urejenih po naraščajočem vrstnem redu glede na njihovo vrednost
          \textit{NOR}.
          Če ima več replik enako vrednost \textit{NOR}, so slednje med seboj
          urejene v padajočem vrstnem redu glede na njihovo velikost. Če so
          tudi velikosti enake, je medsebojni vrstni red teh replik
          naključen. \\
      SizeList & seznam, ki vsebuje velikosti istoležnih replik s seznama
          \textit{ReplicaList} \\
      NORList & seznam, ki vsebuje vrednosti, kolikokrat so bile istoležne
          replike s seznama \textit{ReplicaList} zahtevane s strani
          trenutno opazovanega vozlišča. \\
    \end{tabulary}
  \end{center}

  \caption{Pomen spremenljivk v psevdokodi algoritma MFS.}
  \label{tbl:MFS_vars}
\end{table}

\TODO{force table and to be under the Strategija MFS
section?}

\TODO{strategija Fast Spread with EFS / MFS - kakšne reference na članke, kjer
sta strategiji opisani?}


\section{Implementacija diskretne simulacije}

Za preizkus strategij, opisanih v člankih, sem v programskem jeziku Python
(različica 3.3) napisal diskretno simulacijo. S poganjanjem slednje z
različnimi parametri sem lahko primerjal, kako se obnesejo posamezne strategije
zamenjave. Glavni sestavni deli simulacije so vozlišča, replike, dogodki v
simulaciji in sámo izvajalno okolje simulacije.
\TODO{link na Github, kjer je koda? morda koda kot priloga diplome?}

Topologija podatkovnega omrežja, ki sem jo uporabil, je polni graf, saj so
takšno topologijo izbrali tudi avtorji člankov \TODO{referenca obeh čl.}.
Povezave med vsakim parom vozlišč so simetrične, njihove dolžine pa se v fazi
inicializacije simulacije naključno določijo. V isti fazi se generirajo tudi
replike naključnih velikosti.

Ko so dolžine povezav med vozlišči določene, se na grafu, ki predstavlja
podatkovno omrežje, izvede algoritem Dijkstra, ki poišče najkrajše poti med
glavnim strežnikom in vsemi ostalimi vozlišči. Po izvedbi algoritma vsakemu
vozlišču določimo, kdo je njegov starš na najkrajši poti do glavnega strežnika%
\footnote{Sam glavni strežnik je pri tem seveda izjema, saj se nahaja povsem
na vrhu hierarhije in starša nima.}.

Med izvajanjem simulacije je vsako vozlišče avtonomno in se samo odloča, katere
replike bo hranilo pri sebi, pri čemer sledi vnaprej določeni strategiji
zamenjave. V posamezni izvedbi simulacije vsa vozlišča uporabljajo enako
strategijo in je med izvajanjem ne spreminjajo.

\subsection{Vrste dogodkov v simulaciji}
V okviru simulacije so definirane naslednje štiri vrste dogodkov, ki se lahko
pojavijo v njej:
\begin{itemize}
  \item \textbf{Pošlji zahtevek za repliko:} Vozlišče svojemu staršu pošlje
  zahtevek za repliko, ki je sámo nima. Posledično se za starša po določeni
  zakasnitvi sproži dogodek \textit{prejmi zahtevek za repliko}.

  \item \textbf{Prejmi zahtevek za repliko:} Vozlišče prejme zahtevo za
  določeno repliko bodisi od enega izmed svojih sinov bodisi od zunanjega
  uporabnika podatkovnega omrežja. Če repliko ima, sproži dogodek
  \textit{pošlji repliko}, sicer pa posreduje zahtevo za isto repliko
  naprej svojemu staršu in s tem sproži dogodek \textit{pošlji zahtevek za
  repliko}.

  \item \textbf{Pošlji repliko:} Vozlišče začne pošiljati repliko tja, od
  koder je bila od njega zahtevana. Po določeni zakasnitvi se sproži ustrezen
  dogodek \textit{prejmi repliko}.

  \item \textbf{Prejmi repliko:} Zahtevana replika prispe na mesto, od
  koder je izviral zahtevek zanjo. Če je to mesto izvora izven podatkovnega
  omrežja, se ne zgodi nič več, saj je bil zahtevek zgolj uspešno izpolnjen.
  Če pa gre za vozlišče v omrežju, mora slednje prejeto repliko naprej
  posredovati tja, od koder je bila od njega zahtevana. V tem primeru se
  sproži nov dogodek \textit{pošlji repliko}.
\end{itemize}

Pri dogodku \textit{prejmi zahtevek za repliko} je potrebno omeniti, da se
lahko pojavi tudi takrat, ko je vozlišče isto repliko že zahtevalo in trenutno
čaka, da jo prejme od starša. Ne glede na to vozlišče tudi v tem primeru staršu
pošlje nov zahtevek za isto repliko. Strategiji \ref{alg:EFS} in \ref{alg:MFS}
namreč ne predvidevata, da vozlišče vodi seznam replik, na prejem katerih
že čaka, temveč se vozlišče odloča zgolj na podlagi tega, ali v določenem
trenutku zahtevano repliko ima pri sebi ali ne.

Med izvajanjem simulacije posamezna vozlišča naključno in neodvisno drugo od
drugega sprožajo nove zahtevke za replike. Vse skupaj poteka v izvajalnem
okolju simulacije, ki skrbi za generiranje teh naključnih dogodkov, za
pravilno razporejanje dogodkov v časovni vrsti, za procesiranje dogodkov,
ko pridejo na vrsto, in za beleženje različnih statistik.

Izvajalno okolje skrbi tudi za pravilne zakasnitve dogodkov, ki so že uvrščeni
v časovno vrsto, kadar je to potrebno. Primer tega je dogodek
\textit{pošlji repliko}, ki povzroči, da se trajanje prenosov replik, ki
jih vozlišče že pošilja, podaljša. Replike se namreč pošiljajo sočasno in si
enakomerno delijo pasovno širino, ki je vozlišču na voljo. Posledično mora
izvajalno okolje ob začetku pošiljanja nove replike ustrezno zakasniti vse
dogodke tipa \textit{prejmi repliko}, na katere to vpliva.


\section{Meritve}
\subsection{Priprava simulacije}

Simulacijo sem pripravil na enak način, kot je opisan v obeh člankih
\TODO{referenca?}, in v njej med seboj primerjal štiri strategije zamenjave --
LRU, LFU, EFS in MFS. Za ocenjevanje strategij sem uporabil predlagani
metriki, in sicer skupen odzivni čas in skupno porabo pasovne širine
(Tabela~\ref{tbl:sim_metrics}). Dobra strategija obe metriki minimizira.

Odzivni čas je definiran kot čas, ki preteče od trenutka, ko vozlišče pošlje
zahtevek za repliko, do trenutka, ko zahtevano repliko prejme. Pri tem
predpostavimo, da je čas, potreben za obdelovanje zahtevka in morebitno
zamenjavo replik, zanemarljivo majhen. Če ima vozlišče želeno repliko že
pri sebi, je odzivni čas enak nič.

Porabljena pasovna širina je količina podatkov, ki jih je potrebno prenesti
takrat, kadar vozlišče zahteva repliko, ki je nima shranjene lokalno. Pri
tem se smatra, da je ta količina kar enaka velikosti zahtevane replike.

Ker sta pričakovani vsoti vseh odzivnih časov in velikosti vseh prenešenih
replik zelo veliki, sta obe metriki pomnoženi s konstantama $C_1$ in $C_2$,
ki zmanjšata njuni vrednosti. Vrednost obeh konstant je nastavljena na
$0,001$.
\TODO{dopovej LaTeX-u, da vejica (in ne pika) pomeni decimalno mesto, da bo
pravilno renderiral}

\begin{table}
\small
  \begin{center}
    \begin{tabulary}{1.0\textwidth}{ >{\itshape}p{9em} J}
      \textnormal{Metrika} & Opombe \\
      \hline
      $M_1 = \mathit{TRT} \times C_1$ &
        \textit{TRT}: skupen odzivni čas, $C_1$: konstanta \\
      $M_2 = \mathit{TBC} \times C_2$ &
        \textit{TBC}: skupna porabljena pasovna širina, $C_2$: konstanta
    \end{tabulary}
  \end{center}

  \caption{Metriki za ocenjevanje uspešnosti strategij.%
    \TODO{vir oba članka}}
  \label{tbl:sim_metrics}
\end{table}
\TODO{črta zgoraj nad tabelo? da se bolje ločijo od besedila?}


Replike so kategorizirane v skupine, in sicer tako, da je v vsaki skupini
enako število replik in da vsaka replika pripada zgolj eni skupini.%
\footnote{Kategorizacija replik v skupine ni v nobeni povezavi s tisto skupino
replik, ki jo določeno vozlišče tvori vsakokrat, ko se med izvajanjem
simulacije odloča, katere replike bo morda zamenjalo.}
Število vseh skupin je deset, tako da vsaka izmed njih vsebuje 10~\% replik.

Vsako vozlišče ima svojo ti.~\textit{najbolj želeno skupino} (MWG). Verjetnost,
da vozlišče zahteva določeno repliko, ki pripada njegovi skupini MWG, je višja
od verjetnosti, da zahteva določeno drugo repliko, ki pripada eni izmed
ostalih skupin. To je namreč boljši približek situacijam v praksi,
kjer se dostikrat zgodi, da se določena podmnožica podatkov (replik) v
primerjavi z ostalimi podatki veliko bolj pogosto uporablja.
\TODO{referenca na članek, kjer je to napisano oz. v obeh}

Strategije sem med seboj primerjal v treh različnih scenarijih:
\begin{itemize}
  \item \textbf{Scenarij 1:} Vozlišče zahteva vse replike z
    enako verjetnostjo. To pomeni, da je verjetnost, da zahteva repliko iz
    svoje skupine MWG, zgolj 10~\% -- torej enaka, kot verjetnost za
    katerekoli drugo skupino replik.

  \item \textbf{Scenarij 2:} Vozlišče zahteva eno izmed replik iz svoje
    skupine MWG z verjetnostjo 30~\%. To pomeni, da je verjetnost, da
    zahteva repliko iz ene izmed ostalih skupin, enaka 70~\%.

  \item \textbf{Scenarij 3:} Vozlišče zahteva eno izmed replik iz svoje
    skupine MWG z verjetnostjo 50~\%, z enako verjetnostjo pa tudi repliko
    iz ene izmed ostalih skupin.
\end{itemize}
V vsakem izmed naštetih scenarijev simulacijo poženemo enkrat za vsako
strategijo, kar skupaj znese 12 izvedb simulacije
(3 scenariji $\times$ 4 strategije). V posamezni izvedbi vsa vozlišča
uporabljajo enako strategijo -- tisto, ki jo takrat preizkušamo.

Nastavitve simulacije se med izvedbami ne spreminjajo. Uporabljene vrednosti
parametrov prikazuje Tabela~\ref{tbl:sim_params}.
Velikosti replik, časi med prihodi zahtevkov vozlišč in oddaljenosti med
vozlišči imajo enakomerno diskretno porazdelitev. Zavzamejo lahko katerokoli
celoštevilčno vrednost znotraj njihovega intervala možnih vrednosti.


\begin{table}
\small
  \begin{center}
    \begin{tabulary}{1.0\textwidth}{ >{}p{20em} J}
      \textnormal{Parameter} & Vrednost \\
      \hline
      Število vozlišč & 20 \\
      Število replik & 1000 \\
      Velikost vsake replike & med 100 in 1000 Mbit \\
      Število generiranih zahtevkov & \numprint{100000} \\
      Čas med prihodi zahtevkov vozlišč & med 0 in 99 s \\
      Število skupin replik & 10 \\
      Diskovni prostor na vozliščih (razen glavnega strežnika) &
          \numprint{50000} Mbit \\
      Diskovni prostor na glavnem strežniku & dovolj velik, da lahko naenkrat
          hrani vse replike podatkovnega omrežja
          \TODO{multiline}\\
      Število replik v vsaki skupini & 100 \\
      Pasovna širina omrežja & 10 Mbit/s \\
      Razdalja med vsakim parom neposredno povezanih vozlišč &
        med 1 in 1000 km \\
      Hitrost širjenja signala po mrežnih povezavah & $6 \times 10^3$ km/s \\
      Frequency specific time interval \TODO{prevod?} & \numprint{10000} \\
      C1 & 0,001 \\
      C2 & 0,001
    \end{tabulary}
  \end{center}

  \caption{Parametri simulacije in njihove vrednosti.%
    \TODO{vir oba članka}}
  \label{tbl:sim_params}
\end{table}


\TODO{mogoče tukajali v prilogi v prilogi topologija mreže? torej kakšno drevo
dobimo, kakšne so razdalje med vozlišči}

\TODO{Časovna enota za "inter-arrival times" v člankih ni navedena,
predpostavil sem, da je sekunda. - uporabi "tabu" okolje, ki zna
delati s footnote-i v tabelah? Čeprav nj bi sicer to bila slaba praksa ...
(footnotes v tabelah)}




\subsection{Rezultati simulacij}

Rezultati simulacije so zbrani v Tabeli~\ref{tbl:sim_results}.

\begin{table}
\small
  \begin{center}
    \begin{tabulary}{1.0\textwidth}{ | C | C | C | }
      \hline
      \multicolumn{3}{|>{\bf}c|}{
        Scenarij 1 (P(MWG) = 0.1)
      } \\
      \hline
      Strategija & Skupen odzivni čas [s] & Porabljena pasovna širina [Mbit] \\
      \hline
      LRU & \numprint{55910,66} & \numprint{116780,38} \\
      LFU & \numprint{41760,98} & \numprint{114953,38} \\
      EFS & \numprint{53383,87} & \numprint{116574,68} \\
      MFS & \numprint{53383,87} & \numprint{116574,68} \\
      \hline \hline
      
      \multicolumn{3}{|>{\bf}c|}{
        Scenarij 2 (P(MWG) = 0.3)
      } \\
      \hline
      Strategija & Skupen odzivni čas [s] & Porabljena pasovna širina [Mbit] \\
      \hline
      LRU & \numprint{41550,05} & \numprint{114216,31} \\
      LFU & \numprint{28372,55} & \numprint{109710,82} \\
      EFS & \numprint{39236,63} & \numprint{113760,48} \\
      MFS & \numprint{39236,63} & \numprint{113760,48} \\
      \hline \hline

      \multicolumn{3}{|>{\bf}c|}{
        Scenarij 3 (P(MWG) = 0.5)
      } \\
      \hline
      Strategija & Skupen odzivni čas [s] & Porabljena pasovna širina [Mbit] \\
      \hline
      LRU & \numprint{25356,10} & \numprint{108200,84} \\
      LFU & \numprint{16949,96} & \numprint{98400,73} \\
      EFS & \numprint{24393,25} & \numprint{107298,33} \\
      MFS & \numprint{24393,25} & \numprint{107298,33} \\
      \hline
    \end{tabulary}
    
  \end{center}

  \caption{Rezultati simulacije.%
    \TODO{vir: lastni?}}
  \label{tbl:sim_results}
\end{table}

\TODO{namesto tabele daj raje graf? bo bolj pregledno}

Na koncu povej, da rezultatov iz člankov ni bilo možno ponoviti, zato potem
obstaja sum, da niso nič pametnega naredili, več o tem v nadaljevanju.

Pa kakšen komentar, kaj si ugotovil? padajoči trend is dobil, kar je ok,
ampak predlagani strategiji sta bolj kot LRU, ne pa nekaj izdatno boljšega

\chapter{Kritična presoja člankov}

\TODO{tu pa zdaj cel kup stvari}

\begin{itemize}
\item copy-paste (kako so si podobni ... tega zato zgoraj v uvodu ne pišeš, tam
le omeniš "sum plagiatorstva", tu pa potem razdelaš)

\item meritve se ne ujemajo
Povej, zakaj ... kje je spodnja meja ... kritiziraj metrike ...

\item problemi: nejasnosti, slaba metrika (razdelaj), nappake v algoritmih,
division by zero itd. (pa za čas ni enot ipd. in smo predpostavili sekunde)

\end{itemize}


\chapter{Sklepne ugotovitve}
\TODO{nek zaključek (kakšni šalabajzerji da so) - morda kot zanimivost, da
so pošiljali plagiatorske članke še naprej v recenzijo?}



\chapter*{Sklicevanje na besedilne konstrukte \TODO{delete}}


\label{ch1}
Matematična ali popolna indukcija je eno prvih orodij, ki jih spoznamo za dokazovanje trditev pri matematičnih predmetih. 
\begin{izrek}
\label{iz:1}
Za vsako naravno število $n$ velja
\begin{equation}
n < 2^n.
\label{eq:1}
\end{equation}
\end{izrek}
\begin{dokaz}
Dokazovanje z indukcijo zahteva, da neenakost~\eqref{eq:1} najprej preverimo
za najmanjše naravno število --- $0$. Res, ker je $0 < 1 = 2^0$, je
neenačba~\eqref{eq:1} za $n=0$ izpolnjena.

Sledi indukcijski korak. S predpostavko, da je neenakost~\eqref{eq:1} veljavna
pri nekem naravnem številu $n$, je potrebno pokazati, da je ista neenakost v
veljavi tudi pri njegovem nasledniku --- naravnem številu $n+1$. Računajmo.
\begin{align}
n+1 &< 2^n + 1  \label{eq:2}\\
    &\le 2^n + 2^n \label{eq:3}\\
    &= 2^{n+1} \nonumber
\end{align} 
Neenakost~\eqref{eq:2} je posledica indukcijske predpostavke,
neenakost~\eqref{eq:3} pa enostavno dejstvo, da je za vsako naravno število $n$
izraz $2^n$ vsaj tako velik kot 1. S tem je dokaz Izreka~\ref{iz:1} zaključen.
\end{dokaz}

Opazimo, da je \LaTeX\ številko izreka podredil številki poglavja.



\chapter*{Plovke: slike in tabele \TODO{delete}}
\label{ch2}
Slike in daljše tabele praviloma vključujemo v dokument kot plovke. Pozicija plovke v končnem izdelku ni pogojena s tekom besedila, temveč z izgledom strani. \LaTeX\ bo skušal plovko postaviti samostojno, praviloma na vrh strani, na kateri se na takšno plovko prvič sklicujemo. Pri tem pa bo na vsako stran končnega izdelka želel postaviti tudi sorazmerno velik del besedila. V skrajnem primeru, če imamo res preveč plovk, se bo odločil za stran popolnoma zapolnjeno s plovkami.

\section*{Formati slik}
Bitne slike, vektorske slike, kakršnekoli slike, z \LaTeX{}om lahko vključimo vse. 
Slika~\ref{pic1} je v {\tt .pdf} formatu.
\begin{figure}
\begin{center}
\includegraphics[width=10cm]{pic1.pdf}
\end{center}
\caption{Herschelov graf, vektorska grafika.}
\label{pic1}
\end{figure}
Pa res lahko vključimo slike katerihkoli formatov? Žal ne. Programski paket \LaTeX\ lahko uporabljamo v več dialektih. Ukaz {\tt latex} ne mara vključenih slik v formatu Portable Document Format {\tt .pdf}, ukaz {\tt pdflatex} pa ne prebavi slik v Encapsulated Postscript Formatu {\tt .eps}. 
Strnjeno v Tabeli~\ref{tbl:1}.

\begin{table}
\begin{center}
\begin{tabular}{l|ccc}
ukaz/format & {\tt .pdf} & {\tt .eps} & ostali formati \\ \hline
{\tt pdflatex} & da & ne & da \\
{\tt latex}   & ne & da  & da
\end{tabular}
\end{center}
\caption{}
\label{tbl:1}
\end{table}

Nasvet? Odločite se za uporabo ukaza {\tt pdflatex}. Vaš izdelek bo brez vmesnih stopenj na voljo v {.pdf} formatu in ga lahko odnesete v vsako tiskarno. Če morate na vsak način vključiti sliko, ki jo imate v {\tt .eps} formatu, jo vnaprej pretvorite v alternativni format, denimo {\tt .pdf}.

Včasih se da v okolju za uporabo programskega paketa \LaTeX\ nastaviti na kakšen način bomo prebavljali vhodne dokumente. Spustni meni na Sliki~\ref{pic2} odkriva uporabo \LaTeX{}a v njegovi pdf inkarnaciji --- {\tt pdflatex}.
\begin{figure}
\begin{center}
\includegraphics[width=10cm]{pic2.png}
\end{center}
\caption{Kateri dialekt uporabljati?}
\label{pic2}
\end{figure} 

Vključena Slika~\ref{pic2} je seveda bitna.

Kaj pa stran iz študentskega referata?\label{pp}
Tudi njo lahko vključimo v dokument. Toda ne kot plovko.
 



\begin{thebibliography}{99}
\label{bibliografija}

\bibitem{lf} L.\ Fortnow, ``Viewpoint: Time for computer science to grow up'',
{\it Communications of the ACM}, št.\ 52, zv.\ 8, str.\ 33--35, 2009.
\bibitem{dk1} D.\ E.\ Knuth, P. Bendix. ``Simple word problems in universal algebras'', v zborniku: Computational Problems in Abstract Algebra (ur. J. Leech), 1970, str. 263--297.
\bibitem{lat} L.\ Lamport. {\it LaTEX: A Document Preparation System}. Addison-Wesley, 1986.
\bibitem{bib} O.\ Patashnik (1998) \BibTeX{}ing. 
Dostopno na:\\ http://ftp.univie.ac.at/packages/tex/biblio/bibtex/contrib/doc/btxdoc.pdf
\bibitem{licence} licence-cc.pdf. Dostopno na: 

\end{thebibliography}


\end{document}

