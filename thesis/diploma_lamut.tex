
\documentclass[a4paper, 12pt]{book}

% XXX: why utf8x and not utf8?
\usepackage[utf8x]{inputenc}   % omogoča uporabo slovenskih črk kodiranih v formatu UTF-8 

\usepackage[slovene,english]{babel}  % naloži, med drugim, slovenske delilne vzorce
\usepackage[pdftex]{graphicx}  % omogoča vlaganje slik različnih formatov 
\usepackage{fancyhdr}          % poskrbi, na primer, za glave strani
\usepackage{amssymb}           % dodatni simboli
\usepackage{amsmath}
\usepackage{fixltx2e}          % textsubscript etc. (XXX: used anywhere here?)
\usepackage{color}
\usepackage{array}
\usepackage{tabulary}

\usepackage[lined,plain,linesnumbered]{algorithm2e}
\usepackage{setspace}

% reduce line spacing for algorithms to save space
\usepackage{etoolbox}
\AtBeginEnvironment{algorithm}{\setstretch{1.05}\small}




\renewcommand*{\familydefault}{\rmdefault}  % default font family (roman)
\renewcommand{\baselinestretch}{1.3}  % ustrezen razmik med vrsticami

\renewcommand{\arraystretch}{1.3}  % razmik med vrsticami tabel
\AtBeginEnvironment{tabular}{\small}

%oznake strani
\renewcommand{\chaptermark}[1]%
{\markboth{\MakeUppercase{\thechapter.\ #1}}{}} \renewcommand{\sectionmark}[1]%
{\markright{\MakeUppercase{\thesection.\ #1}}} \renewcommand{\headrulewidth}{0.5pt} \renewcommand{\footrulewidth}{0pt} 
\fancyhf{}
\fancyhead[LE,RO]{\sl \thepage} \fancyhead[LO]{\sl \rightmark} \fancyhead[RE]{\sl \leftmark}

\newcommand{\BibTeX}{{\sc Bib}\TeX}

\newcommand{\autfont}{\Large}
\newcommand{\titfont}{\LARGE\bf}
\newcommand{\newterm}{\textit}

%\newcommand{\TODO}{\textcolor{red}}
\newcommand{\TODO}[1]{\textcolor{red}{TODO: #1}}

\newcommand{\clearemptydoublepage}{\newpage{\pagestyle{empty}\cleardoublepage}}
\setcounter{tocdepth}{2}	      % globina kazala

% konstrukti
\newtheorem{izrek}{Izrek}[chapter]

%\newtheorem{trditev}{Trditev}[izrek]
\newenvironment{dokaz}{\emph{Dokaz.}\ }{\hspace{\fill}{$\Box$}}



\begin{document}
\selectlanguage{slovene}
\frontmatter
\setcounter{page}{1} %
\renewcommand{\thepage}{}       % preprecimo težave s številkami strani v kazalu


%%%%%%%%%%%%%%%%%%%%%%%%%%%%%%%%%%%%%%%%
%naslovnica
\label{naslovnica}
\thispagestyle{empty}%
\begin{center}
    {\large\sc Univerza v Ljubljani\\%
      Fakulteta za računalništvo in informatiko}%
    \vskip 10em%
    {\autfont Peter Lamut\par}%
    {\titfont \TODO{Naslov diplomskega dela} \par}%
    {\vskip 2em \textsc{DIPLOMSKO DELO NA UNIVERZITETNEM ŠTUDIJU
    RAČUNALNIŠTVA IN INFORMATIKE}\par}%
    \vfill\null%
    {\large \textsc{Mentor}: doc.\ dr. Boštjan Slivnik\par}%
    {\vskip 2em \large Ljubljana 2014 \par}%
\end{center}

% prazna stran
\clearemptydoublepage


%%%%%%%%%%%%%%%%%%%%%%%%%%%%%%%%%%%%%%%%
%copyright stran
\label{copyright_page}
\thispagestyle{empty}
\vspace*{8cm}

{\small \noindent
Rezultati diplomskega dela so intelektualna lastnina avtorja in Fakultete za ra\-ču\-nal\-niš\-tvo in informatiko Univerze v Ljubljani. 
Za objavljanje ali izkoriščanje rezultatov di\-plom\-ske\-ga dela je potrebno pisno soglasje avtorja, Fakultete za ra\-ču\-nal\-niš\-tvo in 
informatiko ter mentorja.}%
\footnote{V dogovoru z mentorjem lahko kandidat diplomsko delo s pripadajočo izvorno kodo izda tudi pod katero izmed alternativnih licenc, ki ponuja določen del pravic vsem: npr. Creative Commons, GNU GPL. V tem primeru na to mesto vstavite opis licence, na primer tekst \cite{licence}

 \TODO{Morda pa res raje izdaj pod katero izmed prostih licenc?}
}


\begin{center} 
\mbox{}\vfill
% TODO: a je to potrebno tukaj? kje drugje morda?
\emph{Besedilo je oblikovano z urejevalnikom besedil \LaTeX.} 
\end{center}


% prazna stran
\clearemptydoublepage


%%%%%%%%%%%%%%%%%%%%%%%%%%%%%%%%%%%%%%%%
% stran 3 med uvodnimi listi
\noindent
\TODO{Namesto te strani {\bf vstavite} original izdane teme diplomskega 
dela s podpisom mentorja in dekana ter žigom fakultete, ki ga diplomant
dvigne v študent\-skem referatu,  preden odda izdelek v vezavo!
Glej tudi sam konec Poglavja~\ref{ch2} na strani~\pageref{pp}.}

% prazna stran
\clearemptydoublepage


%%%%%%%%%%%%%%%%%%%%%%%%%%%%%%%%%%%%%%%%
% izjava o avtorstvu
\label{izjava_avtorstvo}

\vspace*{1cm}
\begin{center} 
{\Large \textbf{\sc Izjava o avtorstvu diplomskega dela}}
\end{center}

\vspace{1cm}
\noindent Spodaj podpisani Peter Lamut,
z vpisno številko \textbf{63000200}, sem avtor diplomskega dela z naslovom:
   
\vspace{0.5cm}
\emph{\TODO{naslov diplomskega dela}}

\vspace{1.5cm}
\noindent S svojim podpisom zagotavljam, da:

\begin{itemize}
	\item sem diplomsko delo izdelal samostojno pod mentorstvom 
		doc.\ dr.\ Boštjana Slivnika,

	\item so elektronska oblika diplomskega dela, naslov (slov., angl.), povzetek (slov., angl.) ter ključne besede (slov., angl.) identični s tiskano obliko diplomskega dela,
	
	\item soglašam z javno objavo elektronske oblike diplomskega dela v zbirki ``Dela FRI''.
\end{itemize}

\vspace{1cm}

\noindent V Ljubljani, dne \TODO{datum, npr. 1. aprila 2014} \hfill Podpis avtorja:

% prazna stran
\clearemptydoublepage


%%%%%%%%%%%%%%%%%%%%%%%%%%%%%%%%%%%%%%%%
% zahvala

\label{zahvala}
\thispagestyle{empty}\mbox{}\vfill\null\it%
\TODO{Na tem mestu zapišite, komu se zahvaljujete za izdelavo diplomske naloge. Pazite, da ne boste koga pozabili. Utegnil vam bo zameriti. Temu se da izogniti tako, da pozabite na celo zahvalo.}
\rm\normalfont

% prazna stran
\clearemptydoublepage


%%%%%%%%%%%%%%%%%%%%%%%%%%%%%%%%%%%%%%%%
% kazalo
\label{kazalo}
\def\thepage{}% preprecimo tezave s stevilkami strani v kazalu 
\tableofcontents{}


% prazna stran
\clearemptydoublepage


%%%%%%%%%%%%%%%%%%%%%%%%%%%%%%%%%%%%%%%%
% povzetek
% TODO: napiši na koncu
\addcontentsline{toc}{chapter}{Povzetek}
\chapter*{Povzetek}

\TODO{Ključne besede:}%

\TODO{V vzorcu je predstavljen postopek priprave diplomskega dela z uporabo okolja \LaTeX. Vaš povzetek mora sicer vsebovati približno 100 besed, ta tukaj je odločno prekratek.}
% prazna stran
\clearemptydoublepage


%%%%%%%%%%%%%%%%%%%%%%%%%%%%%%%%%%%%%%%%
% abstract
\selectlanguage{english}
\addcontentsline{toc}{chapter}{Abstract}
\chapter*{Abstract}

\TODO{Keywords:}

\TODO{This sample document presents an approach to typesetting your BSc thesis using \LaTeX. A proper abstract should contain around 100 words which makes this one way too short.}

\selectlanguage{slovene}

% prazna stran
\clearemptydoublepage


%%%%%%%%%%%%%%%%%%%%%%%%%%%%%%%%%%%%%%%%
\mainmatter
\setcounter{page}{1}
\pagestyle{fancy}

\chapter{Uvod}

\section{Splošno o podatkovnih omrežjih}

\TODO{slovenski términ za datagrid pravilen?}

Podatkovno omrežje (angl. \textit{datagrid}) je množica med seboj
povezanih računalnikov na različnih geografskih lokacijah, ki uporabnikom
omogočajo nalaganje, hranjenje in medsebojno izmenjavanje datotek.
(TODO: vir definicije / ustrezno prilagodi definicijo)

\TODO{Omeni, da imaš različne topologije, da je cilj robustnost, redundanca,
itd., sklicuj se pač na vire - odstavek ali dva max. (1 stran)}

\TODO{Pa morda še kakšna slika kot primer (nariši npr. z LaTeXom - kar tako,
 za vajo)}

\section{Replikacija podatkov v podatkovnih omrežjih}

Ko uporabnik podatkovnega omrežja pošlje zahtevek za prejem neke datoteke
ali skupine datotek, se lahko pri prenašanju podatkov od glavnega strežnika
do odjemalca porabi veliko pasovne širine. Poleg tega je lahko tudi čas,
potreben za prenos, dolg. Zaradi tega je morda smiselno za posamezno
datoteko ustvariti več njenih kopij na različnih lokacijah v omrežju.
Tovrstne kopije datotek imenujemo \newterm{replike}.

Cilj ustvarjanja replik je zmanjšati porabo pasovne širine in izboljšati
odzivne čase podatkovnega omrežja (\TODO{vir, kjer sta omenjena ta dva
cilja}). Če je neka replika na voljo tudi lokalno (bliže
uporabniku, ki jo zahteva), je namreč ni potrebno vsakič znova prenašati s
strežnika, temveč se preprosto uporabi lokalna kopija.

Posamezna vozlišča (računalniki) v omrežju praviloma nimajo dovolj prostora,
da bi hranila vse replike naenkrat. Zaradi te omejitve je potrebna
strategija, na podlagi katere se odločimo, katere replike bomo hranili
na katerih vozliščih --- seveda tako, da bo zadani cilj (\TODO{številčna
referenca prej opisanega cilja?}) čim bolje izpolnjen.

Replikacijske strategije ločimo v dve skupini, in sicer poznamo
\newterm{statično replikacijo} in \newterm{dinamično replikacijo}
(\TODO{viri, kjer je to navedeno}).

\subsection{Statična replikacija}

Pri statični replikaciji za vsako posamezno vozlišče že vnaprej določimo,
katere replike bodo shranjene na njem. Problem, kako replike čim bolje
razporediti po omrežju, lahko obravnavamo kot statičen optimizacijski
problem. Zanj se sicer izkaže, da je tako NP-težek kot tudi
neaproksimativen.

\TODO{referenca kakšnega članka, ki o tem govori, npr. Čibej}

\subsection{Dinamična replikacija}

Pri dinamični replikaciji se vsako vozlišče avtonomno odloča, katere
replike bo hranilo, pri čemer se lahko množica shranjenih replik na
posameznem vozlišču skozi čas spreminja. Če vozlišče na podlagi neke
metrike oceni, da je pravkar zahtevana replika zanj bolj pomembna od ene
ali več obstoječih shranjenih replik, lahko slednje izbriše in namesto
njih shrani novo repliko.

Dinamična replikacija je boljša od statične, saj se lahko avtomatično
prilagaja morebitnim spremembam v vzorcih uporabe podatkovnega omrežja
\TODO{citat Čibej}. Poznamo veliko različnih dinamičnih replikacijskih
strategij, pri čemer je ena izmed najbolj učinkovitih algoritem
\newterm{Fast Spread} (\TODO{vir -- Ranganathan and Foster, 2001b?}).

\TODO{Fast spread prevedemo v slovenščino? "algoritem hitrega širjenja"?}


\section{Algoritem Fast Spread}

\TODO{opiši hierarhijo, obstaja najkrajša pot od lista do strežnika in
vsako vozlišče zahteva repliko od starša - do glavnega strežnika, ki
ima dovolj prostora, da hrani vse replike}

Ko neko vozlišče v omrežju prejme zahtevek za določeno repliko in slednje
nima shranjene lokalno, mora zahtevek posredovati enemu izmed sosednjih
vozlišč. To lahko isti zahtevek posreduje še naprej svojim sosedom, dokler
ga ne prejme vozlišče, ki repliko ima. Pri pošiljanju replike nazaj
algoritem Fast Spread repliko shrani na vsakem izmed vozlišč na poti od
ciljnega vozlišča do vozlišča, ki je zahtevek prvotno poslalo.

\TODO{kakšna lepa slikica?}

Če določeno vozlišče nima dovolj prostora, da bi shranilo zahtevano repliko,
mora predhodno izbrisati eno ali več obstoječih replik. Katere izmed
obstoječih replik bodo zamenjane z novo repliko, je odvisno od tega, katero
strategijo zamenjave uporabimo.

\subsection{Strategija zamenjave LRU}

Pri strategiji \newterm{LRU} (\newterm{``Least Recently Used"}) vozlišče
zamenja tisto repliko (ali skupino replik), pri kateri je preteklo največ
časa od prejema zadnjega zahtevka zanjo --- torej tisto, ki najdlje časa
ni bila uporabljena.

\subsection{Strategija zamenjave LFU}

Pri strategiji \newterm{LFU} (\newterm{``Least Frequently Used"}) vozlišče
zamenja tisto repliko (ali skupino replik), ki je bila v preteklosti
najmanjkrat zahtevana. Strategija torej upošteva vse pretekle zahtevke za
posamezno repliko in ne zgolj zadnjega, kot je to pri strategiji LRU.

\section{Trditev v člankih: izboljšava algoritma Fast Spread}

Leta 2011 in 2012 sta bila v dveh različnih strokovnih revijah objavljena
dva članka istih avtorjev, v katerih so slednji predstavili dve novi
strategiji zamenjave, ki naj bi po njihovih trditvah dosegali dosti boljše
rezultate od strategij LRU in LFU. Strategiji
so poimenovali \newterm{EFS} (\newterm{``Enhanced Fast Spread''}) in
\newterm{MFS} (\newterm{``Modified Fast Spread''})
\TODO{vir oz. omemba člankov}.

\TODO{oštevilči članka? Da se kasneje v besedilu sklicuješ nanj? Kako se
to ponavadi dela?}

Članka sta si med seboj že na prvi pogled zelo podobna. Oba imata identično
strukturo (identične naslove posameznih delov članka), vsebujeta številne
identične odseke besedila, identične diagrame in tudi opisa obeh strategij
v psevdokodi sta v veliki meri enaka. Na podlagi naštetega se je pojavil
resen sum na recikliranje člankov (avtoplagiatorstvo).

\TODO{vstavi slike, dele besedila itd. za primerjavo pa to omeni v tekstu}

V okviru diplomskega dela sem z implementacijo strategij, opisanih v
člankih, želel doseči predvsem naslednje cilje:
\label{cilji}

\begin{enumerate}
\item Preveriti rezultate, ki so jih dosegli avtorji člankov.

\item Neposredno primerjati strategiji iz člankov \TODO{2011} in
\TODO{2012} med seboj. V obeh člankih namreč avtorji v njih opisani
strategiji primerjajo zgolj s strategijama LRU in LFU, medsebojne
primerjave pa v novejšem članku (\TODO{2012}) niso naredili.

\item Na podlagi primerjave obeh opisanih strategij ugotoviti, ali med
člankoma sploh obstaja kakšna bistvena vsebinska razlika.
\end{enumerate}


\chapter{Implementacija strategij iz člankov}

\section{Opis strategij zamenjave}
Kot je že bilo omenjeno (\TODO{v poglacju XYZ?}), algoritem Fast Spread
shrani zahtevano repliko na vsakem vozlišču na poti od vozlišča, kjer je
bila zahtevana replika najdena, do vozlišča, ki je repliko prvotno
zahtevalo. Če določeno vozlišče nima dovolj razpoložljivega prostora,
da bi zahtevano repliko shranilo, mora najprej izbrisati eno ali več
obstoječih replik.

Avtorji člankov (\TODO{št. oznaka člankov?}) opozarjajo, da je lahko
skupina replik, ki jih je potrebno izbrisati, ``bolj pomembna" od
nove replike. Izpostavljajo, da algoritem Fast Spread tega ne upošteva in
da zamenjavo skupine obstoječih replik z novo repliko vedno
izvede ne glede na morebitno večjo vrednost skupine replik
(\TODO{sklic na ustrezen del članka - oz. v obeh}).

Avtorji v obeh člankih predlagajo nekoliko spremenjeni različici algoritma
Fast Spread, ki izvedeta zamenjavo le v primerih, ko je vrednost skupine
replik strogo manjša od vrednosti nove replike. Pri tem algoritma za
ocenjevanje vrednosti (skupin) replik uporabljata metrike, opisane v
nadaljevanju.

\subsection{Strategija EFS}

Psevdokoda strategije EFS je prikazana na sliki \TODO{referenca}, pri čemer
je pomen posameznih spremenljivk pojasnjen v tabeli \TODO{referenca}.

\begin{algorithm}
  \label{alg:EFS}
  \caption{Strategija EFS (\TODO{preimenuj bold oznako + referenca na članek})}

  \SetKwComment{blankln}{}{}

  Initialize SOS to $0$\;
  \eIf{RR exists on RN}{
      Use RR\;
  }{
      \For{$i = 2$ \KwTo NSPList.size}{
          \If{RR exists on NSPList(i)}{
              \For{$j = \mathit{NSPList}(i - 1)$ \KwTo $1$}{
                  \eIf{CNFSS $\geq$ RR.Size}{
                      Copy RR\;
                  }{
                      \For{$x = 1$ \KwTo ReplicaList.size}{
                          \eIf{$\mathit{SOS} + \mathit{CNFSS} <$ RR.Size}{
                              SOS = SOS + \textit{SizeList(x)}\;
                          }{
                              Break\;
                          }
                      }

\blankln{}

                      $\mathit{GV} = \frac{\sum_{y=1}^{x-1} \mathit{NORList(y)}}{
                                  \sum_{y=1}^{x-1} \mathit{SizeList(y)}} +
                            \frac{\sum_{y=1}^{x-1}
                              \mathit{NORFSTIList(y)}}{\mathit{FSTI}} +
                            \frac{1}{\mathit{CT} - \frac{\sum_{y=1}^{x-1}
                              \mathit{LRTList(y)}}{x-1}}$

\blankln{}

                      $\mathit{RV_{RR}} = \frac{\mathit{NORRR}}{\mathit{SRR}} +
                                 \frac{\mathit{NORRRFSTI}}{\mathit{FSTI}} +
                                 \frac{1}{\mathit{CT} - \mathit{LRTRR}}$

\blankln{}

                      \If{$\mathit{GV} < \mathit{RV_{RR}}$} {
                          \For{$y = 1$ \KwTo $x-1$}{
                            Delete \textit{ReplicaList(y)}, \textit{NORList(y)},
                            \textit{SizeList(y)}, \textit{NORFSTIList(y)},
                            \textit{LRTList(y)}\;
                          }
                          Copy RR;
                      }
                  }
              }
          }
      }
  }
\end{algorithm}

\begin{table}
\small
  \begin{center}
    \begin{tabulary}{1.0\textwidth}{ >{\itshape}p{7em} J}
      \textnormal{Spremenljivka} & Pomen \\
      \hline
      RR & zahtevana replika \\
      RN &  vozlišče, ki zahteva repliko \\
      CNFSS & nezaseden prostor na trenutno opazovanem vozlišču \\
      FSTI & frekvenčno-specifičen časovni razpon \TODO{prevod???} \\
      NORRR & število zahtevkov za \textit{RR} \\
      SRR & velikost \textit{RR} \\
      NORRRFSTI & število zahtevkov za zahtevano repliko
          v \textit{FSTI} \\
      CT & trenutni čas \\
      LRTRR & čas zadnjega zahtevka za zahtevano repliko \\
      SOS & spremenljivka, ki predstavlja vsoto velikosti skupine replik na
          trenutno opazovanem vozlišču \\
      NSPList & seznam vozlišč na najkrajši poti od \textit{RN} do
          glavnega strežnika \\
      ReplicaList & seznam obstoječih replik na trenutno opazovanem vozlišču,
          urejenih po naraščajočem vrstnem redu glede na njihovo vrednost
          \textit{RV}. Če ima več replik enako vrednost \textit{RV}, so med
          seboj urejene naključno. \\
      NORList & seznam, ki vsebuje vrednosti, kolikokrat so bile istoležne
          replike s seznama \textit{ReplicaList} zahtevane s strani
          trenutno opazovanega vozlišča. \\
      SizeList & seznam, ki vsebuje velikosti istoležnih replik s seznama
          \textit{ReplicaList} \\
      NORFSTIList & seznam, ki vsebuje vrednosti, kolikokrat so bile istoležne
          replike s seznama \textit{ReplicaList} zahtevane v \textit{FSTI} s
          strani trenutno opazovanega vozlišča \\
      LRTList & seznam, ki vsebuje čase zadnjega zahtevka za istoležne
          replike s seznama \textit{ReplicaList}
    \end{tabulary}
  \end{center}

  \caption{Pomen spremenljivk v psevdokodi algoritma EFS.}
  \label{tbl:EFS_vars}
\end{table}

\TODO{kratka pojasnitev algoritma oz. konkretno strategijo menjave
Ko prejme repliko, za shranitev katere nima dovolj prostega prostora, najprej
nabere toliko replik (kandidatk za izbris), da bi brisanje njih sprostilo
dovolj prostora. To je potem skupina replik.
Pri nabiranju kandidatk gre po vrsti od najmanj pomembnih - replike v
ReplicaList so namreč urejene po vrednosti (najmanj pomembne najprej).
\\
Potem pa to naveži naprej, kako EFS oceni vrednost replik in kako skupin replik
- interpretacija formul. pa GV je isto kot RV, samo da vzame povprečja
skupine.
}

Strategija EFS pri ocenjevanju vrednosti posamezne replike upošteva naslednje
štiri stvari:

\begin{itemize}
  \item število zahtevkov za repliko,
  \item pogostost zahtevkov,
  \item velikost replike,
  \item čas od zadnjega zahtevka za repliko.
\end{itemize}

\TODO{item spacing too big}

Število zahtevkov, njihova pogostost in čas od zadnjega zahtevka so pomembni
dejavniki, saj z njihovo pomočjo lahko ocenimo verjetnost, da bo replika
v prihodnosti ponovno zahtevana. Ker imajo vozlišča omejen prostor za
shranjevanje replik, je pomembna tudi velikost slednjih
\TODO{referenca na članek 2011}.

\begin{samepage}
Iz vrstice 19 (\TODO{referenca}) psevdokode \TODO{sklic} lahko razberemo
formulo za izračun vrednosti replike:
\begin{equation}
  \mathit{RV} = \frac{\mathit{NORRR}}{\mathit{SRR}} +
                      \frac{\mathit{NORRRFSTI}}{\mathit{FSTI}} +
                      \frac{1}{\mathit{CT} - \mathit{LRTRR}}
  \label{eq:EFS_RV}
\end{equation}
pri čemer je pomen posameznih spremenljivk enačbe pojasnjen v
Tabeli~\ref{tbl:EFS_vars}.
\end{samepage}

Iz prvega in drugega člena enačbe~\eqref{eq:EFS_RV} je razvidno, da imajo
višjo vrednost tiste replike, ki so bile večkrat zahtevane (večje število
zahtevkov), pri čemer drugi člen upošteva zgolj zahtevke iz nedavne zgodovine,
torej znotraj intervala FSTI, ki jih dodatno normira z velikostjo tega
intervala. Iz prvega člena je tudi razvidno, da je vrednost replik obratno
sorazmerna z njihovo velikostjo. Večje replike so vrednotene niže, saj
zasedajo več prostora na vozlišču. Zadnji člen daje višjo vrednost replikam,
katerih čas od njihovega zadnjega zahtevka je manjši.

\TODO{zdaj pa še group value pojasni}

\begin{samepage}
Podobno kot vrednost posamezne replike se izračuna tudi vrednost skupine
replik, in sicer po formuli iz vrstice 18 \TODO{} v
\TODO{EFS psevdokoda referenca}:
\begin{equation}
  \mathit{GV} = \frac{\sum_{y=1}^{x-1} \mathit{NORList(y)}}{
                      \sum_{y=1}^{x-1} \mathit{SizeList(y)}} +
                \frac{\sum_{y=1}^{x-1} \mathit{NORFSTIList(y)}}{
                      \mathit{FSTI}} +
                \frac{1}{\mathit{CT} - \frac{\sum_{y=1}^{x-1}
                         \mathit{LRTList(y)}}{x-1}}
  \label{eq:EFS_GV}
\end{equation}
pri čemer je pomen posameznih spremenljivk enačbe pojasnjen v
Tabeli~\ref{tbl:EFS_vars}.
\end{samepage}

Enačbi~\eqref{eq:EFS_RV}~in~\eqref{eq:EFS_GV} sta vsebinsko gledano
pravzaprav enaki. Edina razlika je, da so posamezni členi iz
enačbe~\eqref{eq:EFS_RV} v enačbi~\eqref{eq:EFS_GV} namesto za posamezno
repliko izračunani na ravni celotne skupine replik.
Tako je med drugim v prvem členu enačbe~\eqref{eq:EFS_GV} izračunano razmerje
med skupnim številom zahtevkov in skupno velikostjo \textit{vseh replik} iz
skupine, v tretjem členu pa je uporabljen \textit{povprečen čas} od zadnjega
zahtevka za posamezno repliko v skupini.

\TODO{a bi v GV morali deliti 2. člen še s številom replik v skupini?
Konceptualno gledano verjetno da, ampak v članku to ni tako navedeno) ---
na to opozori pri razpravi o članku}

\subsection{Strategija MFS}
To je pa članek 2012 ...


\begin{algorithm}[H]
  \label{alg:MFS}
  \caption{Strategija MFS (\TODO{preimenuj bold oznako + referenca na članek})}

  \SetKwComment{blankln}{}{}

  Initialize SOS to $0$\;
  \eIf{RR exists on RN}{
      Use RR\;
  }{
      \For{$i = 2$ \KwTo NSPList.size}{
          \If{RR exists on NSPList(i)}{
              \For{$j = \mathit{NSPList}(i - 1)$ \KwTo $1$}{
                  \eIf{CNFSS $\geq$ RR.Size}{
                      Copy RR\;
                  }{
                      $\mathit{PNOR} =
                          \mathit{NOR} \times
                          \tfrac{\mathit{RR.Size} - \mathit{CNFSS}}{
                                 \mathit{ RR.Size}}$
                      \For{$x = 1$ \KwTo ReplicaList.size}{
                          \eIf{SOS $<$ RR.Size $- \mathit{CNFSS}$}{
                              SOS = SOS + \textit{SizeList(x)}\;
                          }{
                              Break\;
                          }
                      }

                      \If{$\sum_{y=1}^{x-1} \mathit{NORList(y)} < \mathit{PNOR}$} {
                          \For{$y = 1$ \KwTo $x-1$}{
                              Delete \textit{ReplicaList(y)}, \textit{SizeList(y)},
                              \textit{NORList(y)}\;
                          }
                          Copy RR;
                      }
                  }
              }
          }
      }
  }
\end{algorithm}

Pa še tabela spremenljivk iz psevdokode za MFS

\begin{table}
\small
  \begin{center}
    \begin{tabulary}{1.0\textwidth}{ >{\itshape}p{7em} J}
      \textnormal{Spremenljivka} & Pomen \\
      \hline
      RR & zahtevana replika \\
      RN &  vozlišče, ki zahteva repliko \\
      CNFSS & nezaseden prostor na trenutno opazovanem vozlišču \\
      NOR & število zahtevkov za \textit{RR} \\
      PNOR & delno število zahtevkov za \textit{RR} \\
      SOS & spremenljivka, ki predstavlja vsoto velikosti skupine replik na
          trenutno opazovanem vozlišču \\
      NSPList & seznam vozlišč na najkrajši poti od \textit{RN} do
          glavnega strežnika \\
      ReplicaList & seznam obstoječih replik na trenutno opazovanem vozlišču,
          urejenih po naraščajočem vrstnem redu glede na njihovo vrednost
          \textit{NOR}.
          Če ima več replik enako vrednost \textit{NOR}, so slednje med seboj
          urejene v padajočem vrstnem redu glede na njihovo velikost. Če so
          tudi velikosti enake, je medsebojni vrstni red teh replik
          naključen. \\
      SizeList & seznam, ki vsebuje velikosti istoležnih replik s seznama
          \textit{ReplicaList} \\
      NORList & seznam, ki vsebuje vrednosti, kolikokrat so bile istoležne
          replike s seznama \textit{ReplicaList} zahtevane s strani
          trenutno opazovanega vozlišča. \\
    \end{tabulary}
  \end{center}

  \caption{Pomen spremenljivk v psevdokodi algoritma MFS.}
  \label{tbl:2}
\end{table}


\section{Implementacija}
Povej, kako si naredil Python simulacijo, event SendReplica, ReceiveReplica,
vsak node je neodvisen, kako na začetku poženeš Dijsktro ...

\section{Meritve}
\subsection{Priprava simulacije}
Povej, da imaš štiri strategije (EFS, MFS, LRU, LFU). Pa povej tri scenarije,
z različnim MWG probability. V posameznem scenariju poženemo vsako izmed
štirih opisanih strategij in jih med seboj primerjamo, skupaj 12 poganjanj
(3 scenariji X 4 sstrategije)

Potem na koncu še povej, kakšni so parametri (koliko vozlišč itd. ... neka
tabela). V vsaki rundi seveda enako, razlikuje se le scenarij in pa
uporabljena trategija.

\subsection{Rezultati simulacij}

\TODO{}

Na koncu povej, da rezultatov iz člankov ni bilo možno ponoviti, zato potem
obstaja sum, da niso nič pametnega naredili, več o tem v nadaljevanju.

\chapter{Kritična presoja člankov}

\TODO{tu pa zdaj cel kup stvari}

\begin{itemize}
\item copy-paste (kako so si podobni ... tega zato zgoraj v uvodu ne pišeš, tam
le omeniš "sum plagiatorstva", tu pa potem razdelaš)

\item meritve se ne ujemajo
Povej, zakaj ... kje je spodnja meja ... kritiziraj metrike ...

\item problemi: nejasnosti, slaba metrika (razdelaj), nappake v algoritmih,
division by zero itd.

\end{itemize}


\chapter{Sklepne ugotovitve}
\TODO{nek zaključek (kakšni šalabajzerji da so) - morda kot zanimivost, da
so pošiljali plagiatorske članke še naprej v recenzijo?}



\chapter*{Sklicevanje na besedilne konstrukte \TODO{delete}}


\label{ch1}
Matematična ali popolna indukcija je eno prvih orodij, ki jih spoznamo za dokazovanje trditev pri matematičnih predmetih. 
\begin{izrek}
\label{iz:1}
Za vsako naravno število $n$ velja
\begin{equation}
n < 2^n.
\label{eq:1}
\end{equation}
\end{izrek}
\begin{dokaz}
Dokazovanje z indukcijo zahteva, da neenakost~\eqref{eq:1} najprej preverimo za najmanjše naravno število --- $0$. Res, ker je $0 < 1 = 2^0$, je neenačba~\eqref{eq:1} za $n=0$ izpolnjena.

Sledi indukcijski korak. S predpostavko, da je neenakost~\eqref{eq:1} veljavna pri nekem naravnem številu $n$, je potrebno pokazati, da je ista neenakost v veljavi tudi pri njegovem nasledniku --- naravnem številu $n+1$. Računajmo.
\begin{align}
n+1 &< 2^n + 1  \label{eq:2}\\
    &\le 2^n + 2^n \label{eq:3}\\
    &= 2^{n+1} \nonumber
\end{align} 
Neenakost~\eqref{eq:2} je posledica indukcijske predpostavke, neenakost~\eqref{eq:3} pa enostavno dejstvo, da je za vsako naravno število $n$ izraz $2^n$ vsaj tako velik kot 1. S tem je dokaz Izreka~\ref{iz:1} zaključen.
\end{dokaz}

Opazimo, da je \LaTeX\ številko izreka podredil številki poglavja.



\chapter*{Plovke: slike in tabele \TODO{delete}}
\label{ch2}
Slike in daljše tabele praviloma vključujemo v dokument kot plovke. Pozicija plovke v končnem izdelku ni pogojena s tekom besedila, temveč z izgledom strani. \LaTeX\ bo skušal plovko postaviti samostojno, praviloma na vrh strani, na kateri se na takšno plovko prvič sklicujemo. Pri tem pa bo na vsako stran končnega izdelka želel postaviti tudi sorazmerno velik del besedila. V skrajnem primeru, če imamo res preveč plovk, se bo odločil za stran popolnoma zapolnjeno s plovkami.

\section*{Formati slik}
Bitne slike, vektorske slike, kakršnekoli slike, z \LaTeX{}om lahko vključimo vse. 
Slika~\ref{pic1} je v {\tt .pdf} formatu.
\begin{figure}
\begin{center}
\includegraphics[width=10cm]{pic1.pdf}
\end{center}
\caption{Herschelov graf, vektorska grafika.}
\label{pic1}
\end{figure}
Pa res lahko vključimo slike katerihkoli formatov? Žal ne. Programski paket \LaTeX\ lahko uporabljamo v več dialektih. Ukaz {\tt latex} ne mara vključenih slik v formatu Portable Document Format {\tt .pdf}, ukaz {\tt pdflatex} pa ne prebavi slik v Encapsulated Postscript Formatu {\tt .eps}. 
Strnjeno v Tabeli~\ref{tbl:1}.

\begin{table}
\begin{center}
\begin{tabular}{l|ccc}
ukaz/format & {\tt .pdf} & {\tt .eps} & ostali formati \\ \hline
{\tt pdflatex} & da & ne & da \\
{\tt latex}   & ne & da  & da
\end{tabular}
\end{center}
\caption{}
\label{tbl:1}
\end{table}

Nasvet? Odločite se za uporabo ukaza {\tt pdflatex}. Vaš izdelek bo brez vmesnih stopenj na voljo v {.pdf} formatu in ga lahko odnesete v vsako tiskarno. Če morate na vsak način vključiti sliko, ki jo imate v {\tt .eps} formatu, jo vnaprej pretvorite v alternativni format, denimo {\tt .pdf}.

Včasih se da v okolju za uporabo programskega paketa \LaTeX\ nastaviti na kakšen način bomo prebavljali vhodne dokumente. Spustni meni na Sliki~\ref{pic2} odkriva uporabo \LaTeX{}a v njegovi pdf inkarnaciji --- {\tt pdflatex}.
\begin{figure}
\begin{center}
\includegraphics[width=10cm]{pic2.png}
\end{center}
\caption{Kateri dialekt uporabljati?}
\label{pic2}
\end{figure} 

Vključena Slika~\ref{pic2} je seveda bitna.

Kaj pa stran iz študentskega referata?\label{pp}
Tudi njo lahko vključimo v dokument. Toda ne kot plovko.
 



\begin{thebibliography}{99}
\label{bibliografija}

\bibitem{lf} L.\ Fortnow, ``Viewpoint: Time for computer science to grow up'',
{\it Communications of the ACM}, št.\ 52, zv.\ 8, str.\ 33--35, 2009.
\bibitem{dk1} D.\ E.\ Knuth, P. Bendix. ``Simple word problems in universal algebras'', v zborniku: Computational Problems in Abstract Algebra (ur. J. Leech), 1970, str. 263--297.
\bibitem{lat} L.\ Lamport. {\it LaTEX: A Document Preparation System}. Addison-Wesley, 1986.
\bibitem{bib} O.\ Patashnik (1998) \BibTeX{}ing. 
Dostopno na:\\ http://ftp.univie.ac.at/packages/tex/biblio/bibtex/contrib/doc/btxdoc.pdf
\bibitem{licence} licence-cc.pdf. Dostopno na: 

\end{thebibliography}


\end{document}

